% Options for packages loaded elsewhere
\PassOptionsToPackage{unicode}{hyperref}
\PassOptionsToPackage{hyphens}{url}
%
\documentclass[
  12pt,
]{article}
\usepackage{amsmath,amssymb}
\usepackage{iftex}
\ifPDFTeX
  \usepackage[T1]{fontenc}
  \usepackage[utf8]{inputenc}
  \usepackage{textcomp} % provide euro and other symbols
\else % if luatex or xetex
  \usepackage{unicode-math} % this also loads fontspec
  \defaultfontfeatures{Scale=MatchLowercase}
  \defaultfontfeatures[\rmfamily]{Ligatures=TeX,Scale=1}
\fi
\usepackage{lmodern}
\ifPDFTeX\else
  % xetex/luatex font selection
  \setmainfont[]{Times New Roman}
\fi
% Use upquote if available, for straight quotes in verbatim environments
\IfFileExists{upquote.sty}{\usepackage{upquote}}{}
\IfFileExists{microtype.sty}{% use microtype if available
  \usepackage[]{microtype}
  \UseMicrotypeSet[protrusion]{basicmath} % disable protrusion for tt fonts
}{}
\makeatletter
\@ifundefined{KOMAClassName}{% if non-KOMA class
  \IfFileExists{parskip.sty}{%
    \usepackage{parskip}
  }{% else
    \setlength{\parindent}{0pt}
    \setlength{\parskip}{6pt plus 2pt minus 1pt}}
}{% if KOMA class
  \KOMAoptions{parskip=half}}
\makeatother
\usepackage{xcolor}
\usepackage[margin=1in]{geometry}
\usepackage{color}
\usepackage{fancyvrb}
\newcommand{\VerbBar}{|}
\newcommand{\VERB}{\Verb[commandchars=\\\{\}]}
\DefineVerbatimEnvironment{Highlighting}{Verbatim}{commandchars=\\\{\}}
% Add ',fontsize=\small' for more characters per line
\usepackage{framed}
\definecolor{shadecolor}{RGB}{248,248,248}
\newenvironment{Shaded}{\begin{snugshade}}{\end{snugshade}}
\newcommand{\AlertTok}[1]{\textcolor[rgb]{0.94,0.16,0.16}{#1}}
\newcommand{\AnnotationTok}[1]{\textcolor[rgb]{0.56,0.35,0.01}{\textbf{\textit{#1}}}}
\newcommand{\AttributeTok}[1]{\textcolor[rgb]{0.13,0.29,0.53}{#1}}
\newcommand{\BaseNTok}[1]{\textcolor[rgb]{0.00,0.00,0.81}{#1}}
\newcommand{\BuiltInTok}[1]{#1}
\newcommand{\CharTok}[1]{\textcolor[rgb]{0.31,0.60,0.02}{#1}}
\newcommand{\CommentTok}[1]{\textcolor[rgb]{0.56,0.35,0.01}{\textit{#1}}}
\newcommand{\CommentVarTok}[1]{\textcolor[rgb]{0.56,0.35,0.01}{\textbf{\textit{#1}}}}
\newcommand{\ConstantTok}[1]{\textcolor[rgb]{0.56,0.35,0.01}{#1}}
\newcommand{\ControlFlowTok}[1]{\textcolor[rgb]{0.13,0.29,0.53}{\textbf{#1}}}
\newcommand{\DataTypeTok}[1]{\textcolor[rgb]{0.13,0.29,0.53}{#1}}
\newcommand{\DecValTok}[1]{\textcolor[rgb]{0.00,0.00,0.81}{#1}}
\newcommand{\DocumentationTok}[1]{\textcolor[rgb]{0.56,0.35,0.01}{\textbf{\textit{#1}}}}
\newcommand{\ErrorTok}[1]{\textcolor[rgb]{0.64,0.00,0.00}{\textbf{#1}}}
\newcommand{\ExtensionTok}[1]{#1}
\newcommand{\FloatTok}[1]{\textcolor[rgb]{0.00,0.00,0.81}{#1}}
\newcommand{\FunctionTok}[1]{\textcolor[rgb]{0.13,0.29,0.53}{\textbf{#1}}}
\newcommand{\ImportTok}[1]{#1}
\newcommand{\InformationTok}[1]{\textcolor[rgb]{0.56,0.35,0.01}{\textbf{\textit{#1}}}}
\newcommand{\KeywordTok}[1]{\textcolor[rgb]{0.13,0.29,0.53}{\textbf{#1}}}
\newcommand{\NormalTok}[1]{#1}
\newcommand{\OperatorTok}[1]{\textcolor[rgb]{0.81,0.36,0.00}{\textbf{#1}}}
\newcommand{\OtherTok}[1]{\textcolor[rgb]{0.56,0.35,0.01}{#1}}
\newcommand{\PreprocessorTok}[1]{\textcolor[rgb]{0.56,0.35,0.01}{\textit{#1}}}
\newcommand{\RegionMarkerTok}[1]{#1}
\newcommand{\SpecialCharTok}[1]{\textcolor[rgb]{0.81,0.36,0.00}{\textbf{#1}}}
\newcommand{\SpecialStringTok}[1]{\textcolor[rgb]{0.31,0.60,0.02}{#1}}
\newcommand{\StringTok}[1]{\textcolor[rgb]{0.31,0.60,0.02}{#1}}
\newcommand{\VariableTok}[1]{\textcolor[rgb]{0.00,0.00,0.00}{#1}}
\newcommand{\VerbatimStringTok}[1]{\textcolor[rgb]{0.31,0.60,0.02}{#1}}
\newcommand{\WarningTok}[1]{\textcolor[rgb]{0.56,0.35,0.01}{\textbf{\textit{#1}}}}
\usepackage{longtable,booktabs,array}
\usepackage{calc} % for calculating minipage widths
% Correct order of tables after \paragraph or \subparagraph
\usepackage{etoolbox}
\makeatletter
\patchcmd\longtable{\par}{\if@noskipsec\mbox{}\fi\par}{}{}
\makeatother
% Allow footnotes in longtable head/foot
\IfFileExists{footnotehyper.sty}{\usepackage{footnotehyper}}{\usepackage{footnote}}
\makesavenoteenv{longtable}
\usepackage{graphicx}
\makeatletter
\def\maxwidth{\ifdim\Gin@nat@width>\linewidth\linewidth\else\Gin@nat@width\fi}
\def\maxheight{\ifdim\Gin@nat@height>\textheight\textheight\else\Gin@nat@height\fi}
\makeatother
% Scale images if necessary, so that they will not overflow the page
% margins by default, and it is still possible to overwrite the defaults
% using explicit options in \includegraphics[width, height, ...]{}
\setkeys{Gin}{width=\maxwidth,height=\maxheight,keepaspectratio}
% Set default figure placement to htbp
\makeatletter
\def\fps@figure{htbp}
\makeatother
\setlength{\emergencystretch}{3em} % prevent overfull lines
\providecommand{\tightlist}{%
  \setlength{\itemsep}{0pt}\setlength{\parskip}{0pt}}
\setcounter{secnumdepth}{5}
% definitions for citeproc citations
\NewDocumentCommand\citeproctext{}{}
\NewDocumentCommand\citeproc{mm}{%
  \begingroup\def\citeproctext{#2}\cite{#1}\endgroup}
\makeatletter
 % allow citations to break across lines
 \let\@cite@ofmt\@firstofone
 % avoid brackets around text for \cite:
 \def\@biblabel#1{}
 \def\@cite#1#2{{#1\if@tempswa , #2\fi}}
\makeatother
\newlength{\cslhangindent}
\setlength{\cslhangindent}{1.5em}
\newlength{\csllabelwidth}
\setlength{\csllabelwidth}{3em}
\newenvironment{CSLReferences}[2] % #1 hanging-indent, #2 entry-spacing
 {\begin{list}{}{%
  \setlength{\itemindent}{0pt}
  \setlength{\leftmargin}{0pt}
  \setlength{\parsep}{0pt}
  % turn on hanging indent if param 1 is 1
  \ifodd #1
   \setlength{\leftmargin}{\cslhangindent}
   \setlength{\itemindent}{-1\cslhangindent}
  \fi
  % set entry spacing
  \setlength{\itemsep}{#2\baselineskip}}}
 {\end{list}}
\usepackage{calc}
\newcommand{\CSLBlock}[1]{\hfill\break\parbox[t]{\linewidth}{\strut\ignorespaces#1\strut}}
\newcommand{\CSLLeftMargin}[1]{\parbox[t]{\csllabelwidth}{\strut#1\strut}}
\newcommand{\CSLRightInline}[1]{\parbox[t]{\linewidth - \csllabelwidth}{\strut#1\strut}}
\newcommand{\CSLIndent}[1]{\hspace{\cslhangindent}#1}
\usepackage{tcolorbox}
\usepackage{amssymb}
\usepackage{yfonts}
\usepackage{bm}

\newtcolorbox{greybox}{
  colback=white,
  colframe=blue,
  coltext=black,
  boxsep=5pt,
  arc=4pt}
  
\newcommand{\ds}[4]{\sum_{{#1}=1}^{#3}\sum_{{#2}=1}^{#4}}
\newcommand{\us}[3]{\mathop{\sum\sum}_{1\leq{#2}<{#1}\leq{#3}}}

\newcommand{\ol}[1]{\overline{#1}}
\newcommand{\ul}[1]{\underline{#1}}

\newcommand{\amin}[1]{\mathop{\text{argmin}}_{#1}}
\newcommand{\amax}[1]{\mathop{\text{argmax}}_{#1}}

\newcommand{\ci}{\perp\!\!\!\perp}

\newcommand{\mc}[1]{\mathcal{#1}}
\newcommand{\mb}[1]{\mathbb{#1}}
\newcommand{\mf}[1]{\mathfrak{#1}}

\newcommand{\eps}{\epsilon}
\newcommand{\lbd}{\lambda}
\newcommand{\alp}{\alpha}
\newcommand{\df}{=:}
\newcommand{\am}[1]{\mathop{\text{argmin}}_{#1}}
\newcommand{\ls}[2]{\mathop{\sum\sum}_{#1}^{#2}}
\newcommand{\ijs}{\mathop{\sum\sum}_{1\leq i<j\leq n}}
\newcommand{\jis}{\mathop{\sum\sum}_{1\leq j<i\leq n}}
\newcommand{\sij}{\sum_{i=1}^n\sum_{j=1}^n}
	
\ifLuaTeX
  \usepackage{selnolig}  % disable illegal ligatures
\fi
\IfFileExists{bookmark.sty}{\usepackage{bookmark}}{\usepackage{hyperref}}
\IfFileExists{xurl.sty}{\usepackage{xurl}}{} % add URL line breaks if available
\urlstyle{same}
\hypersetup{
  pdftitle={Interval MDS},
  pdfauthor={Jan de Leeuw - University of California Los Angeles},
  hidelinks,
  pdfcreator={LaTeX via pandoc}}

\title{Interval MDS}
\author{Jan de Leeuw - University of California Los Angeles}
\date{Started November 24 2023, Version of November 27, 2023}

\begin{document}
\maketitle

{
\setcounter{tocdepth}{2}
\tableofcontents
}
\section{Problem}\label{problem}

In the transformation phase of interval MDS we want to minimize
\begin{equation}
\sigma(\alpha,\beta):=\sum_{k=1}^Kw_k(\alpha\delta_k+\beta-d_k)^2
\label{eq:loss}
\end{equation}
over the inequality constraints \(\alpha\geq 0\) and
\(\alpha\delta_k+\beta\geq 0\) and the normalization constraint
\begin{equation}
\sum_{k=1}^Kw_k(\alpha\delta_k+\beta)^2=1.
\end{equation}
The \(w_k\) in definition \eqref{eq:loss} are non-negative weights that
add up to one. Note that some or all of the \(\delta_k\) can be negative.

The theory of normalized cone regression (De Leeuw (1975),
Bauschke, Bui, and Wang (2018), De Leeuw (2019)) shows that we can ignore the
normalization constraint and impose only the inequality constraints. We
find the solution to the normalized problem by normalizing the solution
of the unnormalized problem.

The inequality constraints are obviously equivalent to \(\alpha\geq 0\)
and \(\alpha\delta_{\text{min}}+\beta\geq 0\), with \(\delta_{\text{min}}\)
the smallest of the \(\delta_k\). Now a little change of variables.
Define
\(\gamma:=\alpha\delta_{\text{min}}+\beta\), and let
\(e_k:=\delta_k-\delta_{\text{min}}\). Note that
\(e_k\geq 0\) for all \(k\).

The cone of vectors \(\alpha\delta_k+\beta\) with \(\alpha\geq 0\) and
\(\alpha\delta_k+\beta\geq 0\) is the same as the cone \(\alpha e_k+\gamma\)
with \(\alpha\geq 0\) and \(\gamma\geq 0\). Thus our original (unnormalized)
problem is equivalent to minimization of
\begin{equation}
\sigma(\alpha,\gamma)=\sum_{k=1}^Kw_k(\alpha e_k+\gamma-d_k)^2
\end{equation}
over \(\alpha\geq 0\) and \(\gamma\geq 0\), i.e.~over the non-negative
orthant. Of course the normalization condition becomes
\begin{equation}
\sum_{k=1}^Kw_k(\alpha e_k+\gamma)^2=1.
\end{equation}

\section{Equations}\label{equations}

So let's first consider the unnormalized problem of minimizing
\(\sigma\) over \(\alpha\geq 0\) and \(\gamma\geq 0\).

We start with the trivial observation that the minimum of any function
over the non-negative orthant in two-dimensional space is attained
either in the interior, i.e.~the positive orthant, or on one of the two
rays emanating from the origin along the axes, or at the intersection of
these rays, i.e.~at the origin.

Now, in a more compact notation,
\begin{equation}
\sigma(\alpha,\gamma)=\alpha^2[e^2]+\gamma^2+[d^2]+2\alpha\gamma[e]-2\alpha[ed]-2\gamma[d],
\end{equation}
where the square brackets indicate weighted means. We also assume that the MDS problem is
non-trivial, by which we mean in this context that both \([e]\) and \([d]\) are positive. If \([e]\) is zero all \(\delta_k\) are equal, if \([d]\) is zero all \(d_k\) are zero. Just in case, if all \(e_k\) are zero
then the optimum \(\gamma\) is \(\gamma_0=[d]\), while \(\alpha_0\) is arbitrary. The minimum is equal to
\(\sigma_0=[d^2]-[d]^2\).

If the minimum is attained in the interior then the gradient
must vanish at the minimum. The gradient vanishes at
\begin{align}
\alpha_0&=\frac{[ed]-[e][d]}{[e^2]-[e]^2},\\
\gamma_0&=[d]-\alpha_0[e].
\end{align}
If \(\alpha_0\geq 0\) and \(\gamma_0\geq 0\) we are done and
we have found the required minimum. The minimum
value of \(\sigma\) is
\begin{equation}
\sigma_0=[d^2]-[d]^2-\frac{([ed]-[e][d])^2}{[e^2]-[e]^2}.
\end{equation}

If \((\alpha_0,\gamma_0)\) is not in the non-negative orthant, the minimum
either occurs on the line \(\alpha=0\) or on the line \(\gamma=0\), or at
their intersection.

The minimum on the line \(\alpha=0\) occurs at \(\alpha_1=0\) and \(\gamma_1=[d]\), and is equal
to
\begin{equation}
\sigma_1=\sigma(\alpha_1,\gamma_1)=[d^2]-[d]^2.
\end{equation}
Since \([d]\) is positive the point \((\alpha_1,\gamma_1)\)
is in the non-negative orthant, and thus always feasible. From the normalization
condition we find that the solution of the normalized problem on
\(\alpha=0\) is the point \((0,1)\).

The minimum on the line \(\gamma=0\) occurs at \(\gamma_2=0\) and
\begin{equation}
\alpha_2=\frac{[ed]}{[e^2]},
\end{equation}
and is equal to \begin{equation}
\sigma_2:=\sigma(\alpha_2,0)=[d^2]-\frac{[ed]^2}{[e^2]}.
\end{equation}
Again \(\alpha_2\geq 0\) and thus \((\alpha_2,\gamma_2)\) is always feasible.
the solution to the normalized problem on \(\gamma=0\) is the point
\(([e^2]^{-\frac12}, 0)\).

Thus, summarizing, if \((\alpha_0,\gamma_0)\) is not feasible then the
minimum of the unnormalized problem is at \((\alpha_1,\beta_1)\)
if \(\sigma_1<\sigma_2\) and at
\((\alpha_2,\beta_2)\) if \(\sigma_2<\sigma_1\). There is C code (for the
unweighted case of the smacof project) in the appendix.

\section{Appendix: Code}\label{appendix-code}

\begin{Shaded}
\begin{Highlighting}[]
\PreprocessorTok{\#include }\ImportTok{\textless{}math.h\textgreater{}}
\PreprocessorTok{\#include }\ImportTok{\textless{}stdlib.h\textgreater{}}
\PreprocessorTok{\#define MIN}\OperatorTok{(}\PreprocessorTok{x}\OperatorTok{,}\PreprocessorTok{ y}\OperatorTok{)}\PreprocessorTok{ }\OperatorTok{(((}\PreprocessorTok{x}\OperatorTok{)}\PreprocessorTok{ }\OperatorTok{\textless{}}\PreprocessorTok{ }\OperatorTok{(}\PreprocessorTok{y}\OperatorTok{))}\PreprocessorTok{ }\OperatorTok{?}\PreprocessorTok{ }\OperatorTok{(}\PreprocessorTok{x}\OperatorTok{)}\PreprocessorTok{ }\OperatorTok{:}\PreprocessorTok{ }\OperatorTok{(}\PreprocessorTok{y}\OperatorTok{))}
\PreprocessorTok{\#define SQUARE}\OperatorTok{(}\PreprocessorTok{x}\OperatorTok{)}\PreprocessorTok{ }\OperatorTok{((}\PreprocessorTok{x}\OperatorTok{)}\PreprocessorTok{ }\OperatorTok{*}\PreprocessorTok{ }\OperatorTok{(}\PreprocessorTok{x}\OperatorTok{))}

\DataTypeTok{void}\NormalTok{ smacofUnweightedInterval}\OperatorTok{(}\DataTypeTok{const} \DataTypeTok{unsigned}\NormalTok{ n}\OperatorTok{,} \DataTypeTok{const} \DataTypeTok{double} \OperatorTok{(*}\NormalTok{delta}\OperatorTok{)[}\NormalTok{n}\OperatorTok{][}\NormalTok{n}\OperatorTok{],}
                              \DataTypeTok{const} \DataTypeTok{double} \OperatorTok{(*}\NormalTok{dmat}\OperatorTok{)[}\NormalTok{n}\OperatorTok{][}\NormalTok{n}\OperatorTok{],}
                              \DataTypeTok{double} \OperatorTok{(*}\NormalTok{dhat}\OperatorTok{)[}\NormalTok{n}\OperatorTok{][}\NormalTok{n}\OperatorTok{])} \OperatorTok{\{}
    \DataTypeTok{double}\NormalTok{ deltamin }\OperatorTok{=}\NormalTok{ INFINITY}\OperatorTok{;}
    \ControlFlowTok{for} \OperatorTok{(}\DataTypeTok{unsigned}\NormalTok{ i }\OperatorTok{=} \DecValTok{0}\OperatorTok{;}\NormalTok{ i }\OperatorTok{\textless{}}\NormalTok{ n}\OperatorTok{;}\NormalTok{ i}\OperatorTok{++)} \OperatorTok{\{}
        \ControlFlowTok{for} \OperatorTok{(}\DataTypeTok{unsigned}\NormalTok{ j }\OperatorTok{=} \DecValTok{0}\OperatorTok{;}\NormalTok{ j }\OperatorTok{\textless{}}\NormalTok{ n}\OperatorTok{;}\NormalTok{ j}\OperatorTok{++)} \OperatorTok{\{}
\NormalTok{            deltamin }\OperatorTok{=}\NormalTok{ MIN}\OperatorTok{(}\NormalTok{deltamin}\OperatorTok{,} \OperatorTok{(*}\NormalTok{delta}\OperatorTok{)[}\NormalTok{i}\OperatorTok{][}\NormalTok{j}\OperatorTok{]);}
        \OperatorTok{\}}
    \OperatorTok{\}}
    \DataTypeTok{double}\NormalTok{ sed }\OperatorTok{=} \FloatTok{0.0}\OperatorTok{,}\NormalTok{ see }\OperatorTok{=} \FloatTok{0.0}\OperatorTok{,}\NormalTok{ se }\OperatorTok{=} \FloatTok{0.0}\OperatorTok{,}\NormalTok{ sd }\OperatorTok{=} \FloatTok{0.0}\OperatorTok{,}\NormalTok{ sdd }\OperatorTok{=} \FloatTok{0.0}\OperatorTok{,}
\NormalTok{           dm }\OperatorTok{=} \OperatorTok{(}\DataTypeTok{double}\OperatorTok{)(}\NormalTok{n }\OperatorTok{*} \OperatorTok{(}\NormalTok{n }\OperatorTok{{-}} \DecValTok{1}\OperatorTok{)} \OperatorTok{/} \DecValTok{2}\OperatorTok{);}
    \ControlFlowTok{for} \OperatorTok{(}\DataTypeTok{unsigned}\NormalTok{ j }\OperatorTok{=} \DecValTok{0}\OperatorTok{;}\NormalTok{ j }\OperatorTok{\textless{}} \OperatorTok{(}\NormalTok{n }\OperatorTok{{-}} \DecValTok{1}\OperatorTok{);}\NormalTok{ j}\OperatorTok{++)} \OperatorTok{\{}
        \ControlFlowTok{for} \OperatorTok{(}\DataTypeTok{unsigned}\NormalTok{ i }\OperatorTok{=} \OperatorTok{(}\NormalTok{j }\OperatorTok{+} \DecValTok{1}\OperatorTok{);}\NormalTok{ i }\OperatorTok{\textless{}}\NormalTok{ n}\OperatorTok{;}\NormalTok{ i}\OperatorTok{++)} \OperatorTok{\{}
            \DataTypeTok{double}\NormalTok{ eij }\OperatorTok{=} \OperatorTok{(*}\NormalTok{delta}\OperatorTok{)[}\NormalTok{i}\OperatorTok{][}\NormalTok{j}\OperatorTok{]} \OperatorTok{{-}}\NormalTok{ deltamin}\OperatorTok{;}
            \DataTypeTok{double}\NormalTok{ dij }\OperatorTok{=} \OperatorTok{(*}\NormalTok{dmat}\OperatorTok{)[}\NormalTok{i}\OperatorTok{][}\NormalTok{j}\OperatorTok{];}
\NormalTok{            se }\OperatorTok{+=}\NormalTok{ eij}\OperatorTok{;}
\NormalTok{            sd }\OperatorTok{+=}\NormalTok{ dij}\OperatorTok{;}
\NormalTok{            sed }\OperatorTok{+=}\NormalTok{ eij }\OperatorTok{*}\NormalTok{ dij}\OperatorTok{;}
\NormalTok{            see }\OperatorTok{+=}\NormalTok{ SQUARE}\OperatorTok{(}\NormalTok{eij}\OperatorTok{);}
\NormalTok{            sdd }\OperatorTok{+=}\NormalTok{ SQUARE}\OperatorTok{(}\NormalTok{dij}\OperatorTok{);}
        \OperatorTok{\}}
    \OperatorTok{\}}
\NormalTok{    se }\OperatorTok{/=}\NormalTok{ dm}\OperatorTok{;}
\NormalTok{    sd }\OperatorTok{/=}\NormalTok{ dm}\OperatorTok{;}
\NormalTok{    sed }\OperatorTok{/=}\NormalTok{ dm}\OperatorTok{;}
\NormalTok{    see }\OperatorTok{/=}\NormalTok{ dm}\OperatorTok{;}
\NormalTok{    sdd }\OperatorTok{/=}\NormalTok{ dm}\OperatorTok{;}
    \DataTypeTok{double}\NormalTok{ alpha }\OperatorTok{=} \OperatorTok{(}\NormalTok{sed }\OperatorTok{{-}}\NormalTok{ se }\OperatorTok{*}\NormalTok{ sd}\OperatorTok{)} \OperatorTok{/} \OperatorTok{(}\NormalTok{see }\OperatorTok{{-}}\NormalTok{ SQUARE}\OperatorTok{(}\NormalTok{se}\OperatorTok{));}
    \DataTypeTok{double}\NormalTok{ gamma }\OperatorTok{=} \OperatorTok{(}\NormalTok{sd }\OperatorTok{{-}}\NormalTok{ alpha }\OperatorTok{*}\NormalTok{ se}\OperatorTok{);}
    \ControlFlowTok{if} \OperatorTok{((}\NormalTok{alpha }\OperatorTok{\textless{}} \FloatTok{0.0}\OperatorTok{)} \OperatorTok{||} \OperatorTok{(}\NormalTok{gamma }\OperatorTok{\textless{}} \FloatTok{0.0}\OperatorTok{))} \OperatorTok{\{}
        \DataTypeTok{double}\NormalTok{ s1 }\OperatorTok{=}\NormalTok{ sdd }\OperatorTok{{-}}\NormalTok{ SQUARE}\OperatorTok{(}\NormalTok{sed}\OperatorTok{)} \OperatorTok{/}\NormalTok{ see}\OperatorTok{;}
        \DataTypeTok{double}\NormalTok{ s2 }\OperatorTok{=}\NormalTok{ sdd }\OperatorTok{{-}}\NormalTok{ SQUARE}\OperatorTok{(}\NormalTok{sd}\OperatorTok{);}
        \ControlFlowTok{if} \OperatorTok{(}\NormalTok{s1 }\OperatorTok{\textless{}=}\NormalTok{ s2}\OperatorTok{)} \OperatorTok{\{}
\NormalTok{            alpha }\OperatorTok{=} \FloatTok{0.0}\OperatorTok{;}
\NormalTok{            gamma }\OperatorTok{=}\NormalTok{ sd}\OperatorTok{;}
        \OperatorTok{\}} \ControlFlowTok{else} \OperatorTok{\{}
\NormalTok{            alpha }\OperatorTok{=}\NormalTok{ sed }\OperatorTok{/}\NormalTok{ see}\OperatorTok{;}
\NormalTok{            gamma }\OperatorTok{=} \FloatTok{0.0}\OperatorTok{;}
        \OperatorTok{\}}
    \OperatorTok{\}}
    \DataTypeTok{double}\NormalTok{ beta }\OperatorTok{=}\NormalTok{ gamma }\OperatorTok{{-}}\NormalTok{ alpha }\OperatorTok{*}\NormalTok{ deltamin}\OperatorTok{;}
    \ControlFlowTok{for} \OperatorTok{(}\DataTypeTok{unsigned}\NormalTok{ j }\OperatorTok{=} \DecValTok{0}\OperatorTok{;}\NormalTok{ j }\OperatorTok{\textless{}} \OperatorTok{(}\NormalTok{n }\OperatorTok{{-}} \DecValTok{1}\OperatorTok{);}\NormalTok{ j}\OperatorTok{++)} \OperatorTok{\{}
        \ControlFlowTok{for} \OperatorTok{(}\DataTypeTok{unsigned}\NormalTok{ i }\OperatorTok{=} \OperatorTok{(}\NormalTok{j }\OperatorTok{+} \DecValTok{1}\OperatorTok{);}\NormalTok{ i }\OperatorTok{\textless{}}\NormalTok{ n}\OperatorTok{;}\NormalTok{ i}\OperatorTok{++)} \OperatorTok{\{}
            \OperatorTok{(*}\NormalTok{dhat}\OperatorTok{)[}\NormalTok{i}\OperatorTok{][}\NormalTok{j}\OperatorTok{]} \OperatorTok{=}\NormalTok{ alpha }\OperatorTok{*} \OperatorTok{(*}\NormalTok{delta}\OperatorTok{)[}\NormalTok{i}\OperatorTok{][}\NormalTok{j}\OperatorTok{]} \OperatorTok{+}\NormalTok{ beta}\OperatorTok{;}
            \OperatorTok{(*}\NormalTok{dhat}\OperatorTok{)[}\NormalTok{j}\OperatorTok{][}\NormalTok{i}\OperatorTok{]} \OperatorTok{=} \OperatorTok{(*}\NormalTok{dhat}\OperatorTok{)[}\NormalTok{i}\OperatorTok{][}\NormalTok{j}\OperatorTok{];}
        \OperatorTok{\}}
    \OperatorTok{\}}
    \ControlFlowTok{return}\OperatorTok{;}
\OperatorTok{\}}
\end{Highlighting}
\end{Shaded}

\section*{References}\label{references}
\addcontentsline{toc}{section}{References}

\phantomsection\label{refs}
\begin{CSLReferences}{1}{0}
\bibitem[\citeproctext]{ref-bauschke_bui_wang_18}
Bauschke, H. H., M. N. Bui, and X. Wang. 2018. {``{Projecting onto the Intersection of a Cone and a Sphere}.''} \emph{SIAM Journal on Optimization} 28: 2158--88.

\bibitem[\citeproctext]{ref-deleeuw_U_75a}
De Leeuw, J. 1975. {``{A Normalized Cone Regression Approach to Alternating Least Squares Algorithms}.''} Department of Data Theory FSW/RUL.

\bibitem[\citeproctext]{ref-deleeuw_E_19d}
---------. 2019. {``Normalized Cone Regression.''} 2019. \url{https://jansweb.netlify.app/publication/deleeuw-e-19-d/deleeuw-e-19-d.pdf}.

\end{CSLReferences}

\end{document}
